\documentclass[
  11pt,
  fleqn
]{article}
% [fleqn] left-aligns all equations

% fonts

\usepackage[T1]{fontenc} % Apparently this
% https://tex.stackexchange.com/questions/287312/font-not-loadable-metric-data-not-found-or-bad
\usepackage[full]{textcomp}
\usepackage[final,expansion=alltext,nopatch=footnote]{microtype}
\usepackage{csquotes} % suppress warning from `babel`
\usepackage[english]{babel}
\usepackage[fleqn]{mathtools}
\usepackage[cal=boondoxo]{mathalpha} % mathcal

\usepackage{amsthm} % gives proof environment:
% https://www.overleaf.com/learn/latex/Theorems_and_proofs

\usepackage{newtxtext}
% \usepackage{txfonts} % See
% https://tex.stackexchange.com/questions/444243/symbols-not-showing-up

% These need to be after newtxtext
\usepackage[varqu,varl]{inconsolata} % sans serif typewriter
% \usepackage{cabin} % sans serif - ugly

\usepackage[bigdelims,vvarbb]{newtxmath} % bb from STIX % "should be
% loaded AFTER the text font" -
% https://mirrors.mit.edu/CTAN/fonts/newtx/doc/newtxdoc.pdf

% geometry of the page

\usepackage[
  % top=1in,
  % bottom=1in,
  % left=1in,
  % right=1in
]{geometry}

% paragraph spacing

\setlength{\parindent}{0pt}
\setlength{\parskip}{1.5ex plus 0.4ex minus 0.2ex}

% useful packages

% \usepackage{natbib} % incompatible with BibLaTeX
\usepackage{epsfig}
\usepackage{url}
\usepackage{bm}
\usepackage{blindtext}

% Leon additions

\usepackage{enumitem} % allows for enumerate[resume]

\usepackage{booktabs}

\usepackage[
  style=authoryear-comp,
  natbib=true,
  % refsection=chapter,
  url=false,
  doi=true,
  isbn=false,
  date=year %
  % https://tex.stackexchange.com/questions/55780/disable-month-in-biblatex-bibliography
]{biblatex} % natbib=true allows \citep
% Clear "visited on"
% https://tex.stackexchange.com/questions/400384/how-to-disable-biblatex-showing-visited-on-on-the-references
\AtEveryBibitem{
  \clearfield{urlyear}
  \clearfield{urlmonth}
  \clearfield{eprint} % remove PMID %
  % https://tex.stackexchange.com/questions/250784/removing-eprint-and-eprinttype-in-citation-notes
}

\usepackage{graphicx}
% ensures figures stay in their section
% https://tex.stackexchange.com/questions/279/how-do-i-ensure-that-figures-appear-in-the-section-theyre-associated-with
\usepackage[section]{placeins}

% Section formatting using titlesec
\usepackage{titlesec}
\titleformat*{\section}{\singlespacing\sffamily\Large}
\titleformat*{\subsection}{\singlespacing\sffamily\large}
% \titleformat*{\subsubsection}{\itshape}
\titleformat*{\paragraph}{\itshape}
\titleformat{\chapter}{\singlespacing\sffamily\Huge}{{\thechapter}}{1em}{}

% center figures by default
\makeatletter
\g@addto@macro\@floatboxreset\centering
\makeatother
% italic figure captions
\usepackage[format=plain,
  textfont=it,
]{caption}

% simple TO DO and comment macros
\usepackage[
  % https://tex.stackexchange.com/questions/4503/how-do-i-specify-color-in-rgb-using-hypersetup-in-hyperref
  dvipsnames,svgnames,x11names
]{xcolor}
\newcommand\todo[1]{\textcolor{orange}{[#1]}}% simple TO DO macro
\newcommand\comment[1]{\textcolor{red}{[#1]}}

\usepackage{url}
\usepackage[
  hidelinks,
  pdfa % Required for valid pdf/a output:
  % https://tex.stackexchange.com/questions/431022/pdf-a-validation-problem-the-f-key-is-missing
]{hyperref}
\hypersetup{
  % colorlinks,
  % linkcolor=RoyalBlue4,
  % citecolor=PaleVioletRed4,
  % urlcolor=Firebrick4
}

\newcommand{\indsim}{\overset{\mathrm{ind}}{\sim}}
% https://tex.stackexchange.com/questions/60545/should-i-mathrm-the-d-in-my-integrals
\newcommand*\diff{\mathop{}\!\mathrm{d}}
\newcommand*\Diff[1]{\mathop{}\!\mathrm{d^#1}}
\newcommand{\e}{\mathrm{e}}
\newcommand{\E}{\mathrm{E}}
\renewcommand{\P}{\mathrm{P}}
\newcommand{\N}{\mathrm{N}}
\newcommand{\diag}{\mathrm{diag}}
\newcommand{\ave}{\mathrm{ave}}
\newcommand{\Var}{\mathrm{Var}}
\newcommand{\Cov}{\mathrm{Cov}}
\newcommand{\cov}{\mathrm{cov}}
\newcommand{\sd}{\mathrm{sd}}
\newcommand{\var}{\mathrm{var}}
\newcommand{\corr}{\mathrm{corr}}
\renewcommand\vec{\boldsymbol}

% Overlap-specific
\newcommand{\rank}{R}
\newcommand{\rmax}{m}
\newcommand{\rgap}{r^\text{gap}}
\newcommand{\Ngap}{N_1^\text{gap}}
\newcommand{\pigap}{\pi_1^\text{gap}}

% For appearance of block matrices
% https://tex.stackexchange.com/questions/495903/horizontal-and-vertical-lines-in-pmatrix
\renewcommand{\arraystretch}{1.5}

% Theorem-type environments
\newtheorem{definition}{Definition}[section]
\newtheorem{result}[definition]{Result}
\newtheorem{claim}[definition]{Claim}

% From Datta template
% https://github.com/bibekanandadatta/JHU-Dissertation-Template

\usepackage{setspace}                         % sets space between lines

%%%% JH Library requirement (DO NOT CHANGE)
\def\GlobalMargin{1.0in}                      % margin on all sides
% \def\PrintingOffset{0.5in}                  % additional left
% margin for the printed copy
\def\PrintingOffset{0.0in}
\def\MainTextSpacing{\singlespacing}        % double-spaced main text
% \def\MainTextSpacing{}
\def\CaptionStretch{1.1}                    % to match 2x spacing
% \def\CaptionStretch{1.0}

\captionsetup{font={stretch=\CaptionStretch}}

\geometry{
  letterpaper,
  margin=\GlobalMargin,
  bindingoffset=\PrintingOffset,
  nomarginpar,
  % includehead,
  % headheight=\HeaderHeight,
  % headsep=\HeaderSpace,
  includefoot,
  heightrounded
}

%%%% BIBLIOGRAPHY ITEMS
\def\BibTextSpacing{\onehalfspacing}         % single-spaced bibliography
% \def\BibTextSpacing{\singlespacing}         % single-spaced bibliography
\def\BibItemSpacing{\baselineskip}          % spacing between
% bibliographic items in reference

%%%% UNNUMBERED CHAPTERS, SECTION, and SUBSECTION COMMAND for ADDING to TOC
%% removes the 'Chapter #' title while keeping it listed in the TOC
\newcommand\chap[1]{%
  \chapter*{#1}%
  \markboth{#1}{}
\addcontentsline{toc}{chapter}{#1}}

%% removes the 'Section #' title while keeping it listed in the TOC
\newcommand\sect[1]{%
  \phantomsection
  \section*{#1}%
\addcontentsline{toc}{section}{#1}}

%% removes the 'Subsection #' title while keeping it listed in the TOC
\newcommand\subsect[1]{%
  \phantomsection
  \subsection*{#1}%
\addcontentsline{toc}{subsection}{#1}}

%% removes the 'Subsubsection #' title while keeping it listed in the TOC
\newcommand\subsubsect[1]{%
  \phantomsection
  \subsubsection*{#1}%
\addcontentsline{toc}{subsubsection}{#1}}

% Correct matrix for double spacing
% https://tex.stackexchange.com/questions/137004/matrix-within-equation/137009#137009
\makeatletter
\def\env@matrix{\hskip -\arraycolsep
  \let\@ifnextchar\new@ifnextchar
  \linespread{1}\selectfont
  \renewcommand{\arraystretch}{1.2}%
\array{*\c@MaxMatrixCols c}}
\makeatother

% Footnotes must be 2 points less than main text but larger than 8pt
\usepackage{footmisc}
\renewcommand{\footnotesize}{\fontsize{9pt}{11pt}\selectfont}

% For including CV
\usepackage{pdfpages}

% Generate PDF/A (archival format)
% https://webpages.tuni.fi/latex/pdfa-guide.pdf
\usepackage[a-1b,mathxmp]{pdfx}

\usepackage[
  nottoc % Don't include TOC in TOC
]{tocbibind}

% Redefine `abstract`s so they can be at the start of each chapter
% don't cause the page numbering to reset
% https://tex.stackexchange.com/a/4857
\newenvironment{myAbstract}{
  \rightskip.5in
  \leftskip.5in
  % \itshape
}{}

% For writing/planning purposes: show paragraph level in TOC
\setcounter{secnumdepth}{2}

\addbibresource{bibliography.bib}
\graphicspath{{fig/}}

\title{Guidance on choosing ordinal outcomes and endpoints for clinical trials}
\author{Leon Di Stefano, Katherine Lee, etc\ldots}
\date{\today}

\begin{document}

\maketitle

\tableofcontents

\section{Introduction}

Ordinal outcomes have become increasingly prominent in clinical
resarch, particularly following the COVID-19 pandemic.

Some advantages of ordinal outcomes are that they can
\begin{enumerate}
  \item Provide more information that binary outcomes, including
    composite outcomes, allowing for more efficient and ethical studies
  \item Allow for the incorporation of safety events into one's
    primary outcome and therefore account for many different aspects
    of benefit/harm. At the same time, they do not require a
    fully-specified utility function (which may differ from patient
    to patient or require more extensive development and validation)
  \item Allow for the combination of discrete and continous outcome
    measures (for example a numerical quality of life scale with
    death as the worst outcome)
  \item Allow for the incorporation of intercurrent events into one's
    primary outcome, avoiding complexities of alternative estimand strategies
    like principal stratum estimands (particular for truncating
    events like death) while having more statistical efficiency than
    binary composite estimands (see point (1))
\end{enumerate}

The goal of this manuscript is to serve as a guide for clinicians and trialists
in choosing/developing ordinal outcomes for their studies. We consider both
both the outcome itself (the data recorded on each patient) and effect
measures or ``endpoints'' (the metric used to compare patient groups).

We want to distinguish \emph{choosing} an outcome/effect
measure from \emph{validating} the outcome/effect measure using
external data. This paper is about choosing the outcome and effect
measure. In practice choosing an outcome and effect measure and
validating that outcome and effect measure in pre-trial data will be
an iterative process.

\section{Desiderata for outcomes and effect measures}

We want outcomes that
\begin{enumerate}
  \item reflect patient benefit and harm
  \item capture sufficient information to efficiently discriminate
    among better- and
    worse-off patients.
\end{enumerate}

We want effect measures that
\begin{enumerate}
  \item capture decision-relevant differences in patient benefit and harm
  \item are easy to interpret and to translate into practical recommendations
\end{enumerate}

\section{Techniques for constructing ordinal outcomes}

\subsection{Tie-breaking can be used to fine-grain ordinal outcomes}

Two kinds: crossing, and tie breaking within a level (overriding).

\subsection{Category probabilities affect precision and power mostly via the
probability of the largest category}

\subsection{Ordinal outcomes can incorporate longitudinal or
time-to-event information}

Distinguish between ``state'' and ``trajectory'' variables.

Trajectory variables, including times-to-event, summarize over a timecourse.

These can be incorporated into an ordinal outcome. For example ``worst pSOFA
while hospitalized until day 28''.

Arguably most existing time-to-event analyses are in fact analyses of an
ordinal outcome. The Cox model treats time-to-event as an ordinal outcome:
because (1) the baseline hazard is not modelled, (2) the model itself is
invariant to monotone rescalings of the time axis, and (3) the (partial)
likelihood (ignoring ties) depends only on the ranks of the data (i.e., is a
marginal likelihood).

\section{Guidelines for choosing estimands or effect measures}

\subsection{Model-based versus nonparametric effect measures}

Model-based: common odds ratio; common stopping ratio or common
continuation ratio.

Nonparametric or model-free measures: probabilistic index, win ratio, win odds,
...

Both of these can be considered ``estimands'', insofar as we consider.

Nonparametric effect measures may appear to be assumption-free. However
\begin{itemize}
  \item Assumptions don't need to hold exactly for the model-based
    effect measures to make sense
  \item The model-free methods also may not make sense, if for
    example there are nonmonotone effects across cutpoints
  \item Model-based methods have additional advantages, for example
    the possibility of adjustment for covariates.

    When additivity approximately holds, this yields a ``best of both
    worlds'' effect measure that is
    personalized effect measure that is yet reportable as a single
    number. Risk ratio for vaccine efficacy (being approximately
    constant as a function of baseline risk/age) as an example.
\end{itemize}

Simulations show a typically very strong association between
proportional odds ratios and probabilistic indices [cite Harrell].

Reducing to a composite binary outcome
\begin{itemize}
  \item Based on a specific cutpoint, e.g. ``hospitalization or worse''
  \item Based on a quantile in a reference group; for example,
    ``worse than the median outcome under standard care''
\end{itemize}

One point of view is that one would ``ideally'' analyze ordinal
outcomes by reporting the associated expected utilities [cite Berry
talk; example of utility-weighted Rankin scale].

\subsection{Generalized pairwise comparisons (``win'' methods)
  are implicitly about constructing
ordinal outcomes, in a way that is compatible with right-censoring}

\end{document}
